% Options for packages loaded elsewhere
\PassOptionsToPackage{unicode}{hyperref}
\PassOptionsToPackage{hyphens}{url}
%
\documentclass[
]{article}
\usepackage{amsmath,amssymb}
\usepackage{lmodern}
\usepackage{iftex}
\ifPDFTeX
  \usepackage[T1]{fontenc}
  \usepackage[utf8]{inputenc}
  \usepackage{textcomp} % provide euro and other symbols
\else % if luatex or xetex
  \usepackage{unicode-math}
  \defaultfontfeatures{Scale=MatchLowercase}
  \defaultfontfeatures[\rmfamily]{Ligatures=TeX,Scale=1}
\fi
% Use upquote if available, for straight quotes in verbatim environments
\IfFileExists{upquote.sty}{\usepackage{upquote}}{}
\IfFileExists{microtype.sty}{% use microtype if available
  \usepackage[]{microtype}
  \UseMicrotypeSet[protrusion]{basicmath} % disable protrusion for tt fonts
}{}
\makeatletter
\@ifundefined{KOMAClassName}{% if non-KOMA class
  \IfFileExists{parskip.sty}{%
    \usepackage{parskip}
  }{% else
    \setlength{\parindent}{0pt}
    \setlength{\parskip}{6pt plus 2pt minus 1pt}}
}{% if KOMA class
  \KOMAoptions{parskip=half}}
\makeatother
\usepackage{xcolor}
\usepackage[margin=1in]{geometry}
\usepackage{color}
\usepackage{fancyvrb}
\newcommand{\VerbBar}{|}
\newcommand{\VERB}{\Verb[commandchars=\\\{\}]}
\DefineVerbatimEnvironment{Highlighting}{Verbatim}{commandchars=\\\{\}}
% Add ',fontsize=\small' for more characters per line
\usepackage{framed}
\definecolor{shadecolor}{RGB}{248,248,248}
\newenvironment{Shaded}{\begin{snugshade}}{\end{snugshade}}
\newcommand{\AlertTok}[1]{\textcolor[rgb]{0.94,0.16,0.16}{#1}}
\newcommand{\AnnotationTok}[1]{\textcolor[rgb]{0.56,0.35,0.01}{\textbf{\textit{#1}}}}
\newcommand{\AttributeTok}[1]{\textcolor[rgb]{0.77,0.63,0.00}{#1}}
\newcommand{\BaseNTok}[1]{\textcolor[rgb]{0.00,0.00,0.81}{#1}}
\newcommand{\BuiltInTok}[1]{#1}
\newcommand{\CharTok}[1]{\textcolor[rgb]{0.31,0.60,0.02}{#1}}
\newcommand{\CommentTok}[1]{\textcolor[rgb]{0.56,0.35,0.01}{\textit{#1}}}
\newcommand{\CommentVarTok}[1]{\textcolor[rgb]{0.56,0.35,0.01}{\textbf{\textit{#1}}}}
\newcommand{\ConstantTok}[1]{\textcolor[rgb]{0.00,0.00,0.00}{#1}}
\newcommand{\ControlFlowTok}[1]{\textcolor[rgb]{0.13,0.29,0.53}{\textbf{#1}}}
\newcommand{\DataTypeTok}[1]{\textcolor[rgb]{0.13,0.29,0.53}{#1}}
\newcommand{\DecValTok}[1]{\textcolor[rgb]{0.00,0.00,0.81}{#1}}
\newcommand{\DocumentationTok}[1]{\textcolor[rgb]{0.56,0.35,0.01}{\textbf{\textit{#1}}}}
\newcommand{\ErrorTok}[1]{\textcolor[rgb]{0.64,0.00,0.00}{\textbf{#1}}}
\newcommand{\ExtensionTok}[1]{#1}
\newcommand{\FloatTok}[1]{\textcolor[rgb]{0.00,0.00,0.81}{#1}}
\newcommand{\FunctionTok}[1]{\textcolor[rgb]{0.00,0.00,0.00}{#1}}
\newcommand{\ImportTok}[1]{#1}
\newcommand{\InformationTok}[1]{\textcolor[rgb]{0.56,0.35,0.01}{\textbf{\textit{#1}}}}
\newcommand{\KeywordTok}[1]{\textcolor[rgb]{0.13,0.29,0.53}{\textbf{#1}}}
\newcommand{\NormalTok}[1]{#1}
\newcommand{\OperatorTok}[1]{\textcolor[rgb]{0.81,0.36,0.00}{\textbf{#1}}}
\newcommand{\OtherTok}[1]{\textcolor[rgb]{0.56,0.35,0.01}{#1}}
\newcommand{\PreprocessorTok}[1]{\textcolor[rgb]{0.56,0.35,0.01}{\textit{#1}}}
\newcommand{\RegionMarkerTok}[1]{#1}
\newcommand{\SpecialCharTok}[1]{\textcolor[rgb]{0.00,0.00,0.00}{#1}}
\newcommand{\SpecialStringTok}[1]{\textcolor[rgb]{0.31,0.60,0.02}{#1}}
\newcommand{\StringTok}[1]{\textcolor[rgb]{0.31,0.60,0.02}{#1}}
\newcommand{\VariableTok}[1]{\textcolor[rgb]{0.00,0.00,0.00}{#1}}
\newcommand{\VerbatimStringTok}[1]{\textcolor[rgb]{0.31,0.60,0.02}{#1}}
\newcommand{\WarningTok}[1]{\textcolor[rgb]{0.56,0.35,0.01}{\textbf{\textit{#1}}}}
\usepackage{graphicx}
\makeatletter
\def\maxwidth{\ifdim\Gin@nat@width>\linewidth\linewidth\else\Gin@nat@width\fi}
\def\maxheight{\ifdim\Gin@nat@height>\textheight\textheight\else\Gin@nat@height\fi}
\makeatother
% Scale images if necessary, so that they will not overflow the page
% margins by default, and it is still possible to overwrite the defaults
% using explicit options in \includegraphics[width, height, ...]{}
\setkeys{Gin}{width=\maxwidth,height=\maxheight,keepaspectratio}
% Set default figure placement to htbp
\makeatletter
\def\fps@figure{htbp}
\makeatother
\setlength{\emergencystretch}{3em} % prevent overfull lines
\providecommand{\tightlist}{%
  \setlength{\itemsep}{0pt}\setlength{\parskip}{0pt}}
\setcounter{secnumdepth}{-\maxdimen} % remove section numbering
\ifLuaTeX
  \usepackage{selnolig}  % disable illegal ligatures
\fi
\IfFileExists{bookmark.sty}{\usepackage{bookmark}}{\usepackage{hyperref}}
\IfFileExists{xurl.sty}{\usepackage{xurl}}{} % add URL line breaks if available
\urlstyle{same} % disable monospaced font for URLs
\hypersetup{
  hidelinks,
  pdfcreator={LaTeX via pandoc}}

\author{}
\date{\vspace{-2.5em}}

\begin{document}

\begin{Shaded}
\begin{Highlighting}[]
\FunctionTok{library}\NormalTok{(tidyverse)}
\end{Highlighting}
\end{Shaded}

\begin{verbatim}
## -- Attaching core tidyverse packages ------------------------ tidyverse 2.0.0 --
## v dplyr     1.1.0     v readr     2.1.4
## v forcats   1.0.0     v stringr   1.5.0
## v ggplot2   3.4.1     v tibble    3.2.0
## v lubridate 1.9.2     v tidyr     1.3.0
## v purrr     1.0.1     
## -- Conflicts ------------------------------------------ tidyverse_conflicts() --
## x dplyr::filter() masks stats::filter()
## x dplyr::lag()    masks stats::lag()
## i Use the ]8;;http://conflicted.r-lib.org/conflicted package]8;; to force all conflicts to become errors
\end{verbatim}

\begin{Shaded}
\begin{Highlighting}[]
\FunctionTok{library}\NormalTok{(rhdf5)}
\end{Highlighting}
\end{Shaded}

\begin{Shaded}
\begin{Highlighting}[]
\NormalTok{file }\OtherTok{=} \StringTok{"STA130\_APOGEE.h5"}
\end{Highlighting}
\end{Shaded}

Research Question 1: Are red giants with higher intensity measurements
harder to measure?

Extracting relevant columns from the dataset:

\begin{Shaded}
\begin{Highlighting}[]
\NormalTok{spectra\_1 }\OtherTok{\textless{}{-}}\NormalTok{ file }\SpecialCharTok{\%\textgreater{}\%} \FunctionTok{h5read}\NormalTok{(}\StringTok{"spectra"}\NormalTok{, }\AttributeTok{index=}\FunctionTok{list}\NormalTok{(}\ConstantTok{NULL}\NormalTok{, }\DecValTok{1}\SpecialCharTok{:}\DecValTok{30000}\NormalTok{)) }\SpecialCharTok{\%\textgreater{}\%} \FunctionTok{t}\NormalTok{() }\SpecialCharTok{\%\textgreater{}\%} \FunctionTok{as\_tibble}\NormalTok{()}
\end{Highlighting}
\end{Shaded}

\begin{verbatim}
## Warning: The `x` argument of `as_tibble.matrix()` must have unique column names if
## `.name_repair` is omitted as of tibble 2.0.0.
## i Using compatibility `.name_repair`.
\end{verbatim}

\begin{Shaded}
\begin{Highlighting}[]
\NormalTok{spectra\_2 }\OtherTok{\textless{}{-}}\NormalTok{ file }\SpecialCharTok{\%\textgreater{}\%} \FunctionTok{h5read}\NormalTok{(}\StringTok{"spectra"}\NormalTok{, }\AttributeTok{index=}\FunctionTok{list}\NormalTok{(}\ConstantTok{NULL}\NormalTok{, }\DecValTok{30001}\SpecialCharTok{:}\DecValTok{60000}\NormalTok{)) }\SpecialCharTok{\%\textgreater{}\%} \FunctionTok{t}\NormalTok{() }\SpecialCharTok{\%\textgreater{}\%} \FunctionTok{as\_tibble}\NormalTok{()}
\NormalTok{spectra\_3 }\OtherTok{\textless{}{-}}\NormalTok{ file }\SpecialCharTok{\%\textgreater{}\%} \FunctionTok{h5read}\NormalTok{(}\StringTok{"spectra"}\NormalTok{, }\AttributeTok{index=}\FunctionTok{list}\NormalTok{(}\ConstantTok{NULL}\NormalTok{, }\DecValTok{60001}\SpecialCharTok{:}\DecValTok{99705}\NormalTok{)) }\SpecialCharTok{\%\textgreater{}\%} \FunctionTok{t}\NormalTok{() }\SpecialCharTok{\%\textgreater{}\%} \FunctionTok{as\_tibble}\NormalTok{()}

\NormalTok{snr }\OtherTok{\textless{}{-}}\NormalTok{ file }\SpecialCharTok{\%\textgreater{}\%} \FunctionTok{h5read}\NormalTok{(}\StringTok{"snr"}\NormalTok{) }\SpecialCharTok{\%\textgreater{}\%} \FunctionTok{as\_tibble}\NormalTok{()}
\end{Highlighting}
\end{Shaded}

Cleaning the data:

\begin{Shaded}
\begin{Highlighting}[]
\NormalTok{spectra\_means\_1 }\OtherTok{\textless{}{-}} \FunctionTok{apply}\NormalTok{(spectra\_1, }\DecValTok{1}\NormalTok{, mean) }\SpecialCharTok{\%\textgreater{}\%} \FunctionTok{as\_tibble}\NormalTok{()}
\NormalTok{spectra\_means\_2 }\OtherTok{\textless{}{-}} \FunctionTok{apply}\NormalTok{(spectra\_2, }\DecValTok{1}\NormalTok{, mean) }\SpecialCharTok{\%\textgreater{}\%} \FunctionTok{as\_tibble}\NormalTok{()}
\NormalTok{spectra\_means\_3 }\OtherTok{\textless{}{-}} \FunctionTok{apply}\NormalTok{(spectra\_3, }\DecValTok{1}\NormalTok{, mean) }\SpecialCharTok{\%\textgreater{}\%} \FunctionTok{as\_tibble}\NormalTok{()}
\NormalTok{spectra\_means }\OtherTok{\textless{}{-}} \FunctionTok{rbind}\NormalTok{(spectra\_means\_1, spectra\_means\_2, spectra\_means\_3)}
\end{Highlighting}
\end{Shaded}

Analysis:

\begin{Shaded}
\begin{Highlighting}[]
\FunctionTok{lm}\NormalTok{(snr}\SpecialCharTok{$}\NormalTok{value }\SpecialCharTok{\textasciitilde{}}\NormalTok{ spectra\_means}\SpecialCharTok{$}\NormalTok{value) }\SpecialCharTok{\%\textgreater{}\%} \FunctionTok{summary}\NormalTok{()}
\end{Highlighting}
\end{Shaded}

\begin{verbatim}
## 
## Call:
## lm(formula = snr$value ~ spectra_means$value)
## 
## Residuals:
##     Min      1Q  Median      3Q     Max 
## -316.23 -135.47  -65.41   67.39 2565.68 
## 
## Coefficients:
##                     Estimate Std. Error t value Pr(>|t|)    
## (Intercept)          2681.17      45.03   59.54   <2e-16 ***
## spectra_means$value -2442.06      47.01  -51.94   <2e-16 ***
## ---
## Signif. codes:  0 '***' 0.001 '**' 0.01 '*' 0.05 '.' 0.1 ' ' 1
## 
## Residual standard error: 205.4 on 99703 degrees of freedom
## Multiple R-squared:  0.02635,    Adjusted R-squared:  0.02634 
## F-statistic:  2698 on 1 and 99703 DF,  p-value: < 2.2e-16
\end{verbatim}

\begin{Shaded}
\begin{Highlighting}[]
\NormalTok{mod }\OtherTok{=} \FunctionTok{lm}\NormalTok{(snr}\SpecialCharTok{$}\NormalTok{value }\SpecialCharTok{\textasciitilde{}}\NormalTok{ spectra\_means}\SpecialCharTok{$}\NormalTok{value)}
\NormalTok{res }\OtherTok{=} \FunctionTok{summary}\NormalTok{(mod)}

\FunctionTok{pt}\NormalTok{(}\FunctionTok{coef}\NormalTok{(res)[, }\DecValTok{3}\NormalTok{], mod}\SpecialCharTok{$}\NormalTok{df, }\AttributeTok{lower =} \ConstantTok{TRUE}\NormalTok{)}
\end{Highlighting}
\end{Shaded}

\begin{verbatim}
##         (Intercept) spectra_means$value 
##                   1                   0
\end{verbatim}

Visualizations:

\begin{Shaded}
\begin{Highlighting}[]
\FunctionTok{ggplot}\NormalTok{() }\SpecialCharTok{+} \FunctionTok{aes}\NormalTok{(}\AttributeTok{x=}\NormalTok{spectra\_means}\SpecialCharTok{$}\NormalTok{value, }\AttributeTok{y=}\NormalTok{snr}\SpecialCharTok{$}\NormalTok{value) }\SpecialCharTok{+} \FunctionTok{geom\_point}\NormalTok{() }\SpecialCharTok{+} \FunctionTok{geom\_smooth}\NormalTok{(}\AttributeTok{method=}\StringTok{"lm"}\NormalTok{, }\AttributeTok{se=}\ConstantTok{FALSE}\NormalTok{) }\SpecialCharTok{+}
  \FunctionTok{labs}\NormalTok{(}\AttributeTok{title=}\StringTok{"Scatter Plot of Average Intensity Measurements against Snr"}\NormalTok{, }\AttributeTok{x=}\StringTok{"Average Intensity Measurements"}\NormalTok{, }\AttributeTok{y=}\StringTok{"Signal{-}to{-}noise Ratio"}\NormalTok{)}
\end{Highlighting}
\end{Shaded}

\begin{verbatim}
## `geom_smooth()` using formula = 'y ~ x'
\end{verbatim}

\includegraphics{Q1_files/figure-latex/unnamed-chunk-6-1.pdf}

\begin{Shaded}
\begin{Highlighting}[]
\FunctionTok{ggplot}\NormalTok{() }\SpecialCharTok{+} \FunctionTok{aes}\NormalTok{(}\AttributeTok{x=}\NormalTok{spectra\_means}\SpecialCharTok{$}\NormalTok{value, }\AttributeTok{y=}\NormalTok{snr}\SpecialCharTok{$}\NormalTok{value) }\SpecialCharTok{+} \FunctionTok{geom\_quantile}\NormalTok{() }\SpecialCharTok{+} \FunctionTok{labs}\NormalTok{(}\AttributeTok{title=}\StringTok{"Quantile Plot of Average Intensity Measurements against Snr"}\NormalTok{, }\AttributeTok{x=}\StringTok{"Average Intensity Measurements"}\NormalTok{, }\AttributeTok{y=}\StringTok{"Signal{-}to{-}noise Ratio"}\NormalTok{)}
\end{Highlighting}
\end{Shaded}

\begin{verbatim}
## Smoothing formula not specified. Using: y ~ x
\end{verbatim}

\includegraphics{Q1_files/figure-latex/unnamed-chunk-6-2.pdf}

Research Question 2: Does surface gravity vary by a small factor among
different red giants?

Extracting the relevant columns:

\begin{Shaded}
\begin{Highlighting}[]
\NormalTok{logg }\OtherTok{\textless{}{-}}\NormalTok{ file }\SpecialCharTok{\%\textgreater{}\%} \FunctionTok{h5read}\NormalTok{(}\StringTok{"logg"}\NormalTok{) }\SpecialCharTok{\%\textgreater{}\%} \FunctionTok{as\_tibble}\NormalTok{()}
\end{Highlighting}
\end{Shaded}

Analysis:

\begin{Shaded}
\begin{Highlighting}[]
\NormalTok{logg\_original\_sample }\OtherTok{\textless{}{-}}\NormalTok{ logg }\SpecialCharTok{\%\textgreater{}\%} \FunctionTok{sample\_n}\NormalTok{(}\AttributeTok{size=}\DecValTok{10000}\NormalTok{)}

\NormalTok{sample\_cov }\OtherTok{\textless{}{-}} \FunctionTok{rep}\NormalTok{(}\ConstantTok{NA}\NormalTok{,}\DecValTok{1000}\NormalTok{)}
\ControlFlowTok{for}\NormalTok{(i }\ControlFlowTok{in} \DecValTok{1}\SpecialCharTok{:}\DecValTok{1000}\NormalTok{)\{}
\NormalTok{  logg10000 }\OtherTok{\textless{}{-}}\NormalTok{ logg\_original\_sample }\SpecialCharTok{\%\textgreater{}\%} \FunctionTok{sample\_n}\NormalTok{(}\AttributeTok{size=}\DecValTok{10000}\NormalTok{,}\AttributeTok{replace =} \ConstantTok{TRUE}\NormalTok{)}
\NormalTok{  sample\_cov[i]}\OtherTok{\textless{}{-}} \FunctionTok{as.numeric}\NormalTok{(logg10000 }\SpecialCharTok{\%\textgreater{}\%} \FunctionTok{summarise}\NormalTok{(}\FunctionTok{sd}\NormalTok{(value))) }\SpecialCharTok{/} \FunctionTok{as.numeric}\NormalTok{(logg10000 }\SpecialCharTok{\%\textgreater{}\%} \FunctionTok{summarise}\NormalTok{(}\FunctionTok{mean}\NormalTok{(value)))}
\NormalTok{\}}
\NormalTok{logg\_cov }\OtherTok{\textless{}{-}} \FunctionTok{tibble}\NormalTok{(}\AttributeTok{coeff\_of\_var =}\NormalTok{ sample\_cov)}
 
\FunctionTok{quantile}\NormalTok{(logg\_cov}\SpecialCharTok{$}\NormalTok{coeff\_of\_var,}\FunctionTok{c}\NormalTok{(}\FloatTok{0.025}\NormalTok{,}\FloatTok{0.975}\NormalTok{))}
\end{Highlighting}
\end{Shaded}

\begin{verbatim}
##      2.5%     97.5% 
## 0.2244030 0.2316455
\end{verbatim}

Visualizations:

\begin{Shaded}
\begin{Highlighting}[]
\NormalTok{logg\_cov }\SpecialCharTok{\%\textgreater{}\%} \FunctionTok{ggplot}\NormalTok{() }\SpecialCharTok{+} \FunctionTok{aes}\NormalTok{(}\AttributeTok{x =}\NormalTok{ coeff\_of\_var) }\SpecialCharTok{+} 
  \FunctionTok{geom\_histogram}\NormalTok{(}\AttributeTok{bins=}\DecValTok{30}\NormalTok{, }\AttributeTok{colour =} \StringTok{\textquotesingle{}black\textquotesingle{}}\NormalTok{, }\AttributeTok{fill =} \StringTok{"grey"}\NormalTok{) }\SpecialCharTok{+} 
  \FunctionTok{geom\_vline}\NormalTok{(}\FunctionTok{aes}\NormalTok{(}\AttributeTok{xintercept =} \FloatTok{0.2236308}\NormalTok{), }\AttributeTok{color=}\StringTok{\textquotesingle{}red\textquotesingle{}}\NormalTok{) }\SpecialCharTok{+} 
  \FunctionTok{geom\_vline}\NormalTok{(}\FunctionTok{aes}\NormalTok{(}\AttributeTok{xintercept =} \FloatTok{0.2308553}\NormalTok{), }\AttributeTok{color=}\StringTok{\textquotesingle{}red\textquotesingle{}}\NormalTok{) }\SpecialCharTok{+}
  \FunctionTok{labs}\NormalTok{(}\AttributeTok{title=}\StringTok{\textquotesingle{}Histogram of Distribution of Coefficient of Variation over 1000 bootstrap samples}
\StringTok{  of size 10000. The red lines show the 95\% confidence interval.\textquotesingle{}}\NormalTok{,}
       \AttributeTok{x=}\StringTok{\textquotesingle{}Coefficient of Variation\textquotesingle{}}\NormalTok{, }\AttributeTok{y=}\StringTok{\textquotesingle{}Count\textquotesingle{}}\NormalTok{)}
\end{Highlighting}
\end{Shaded}

\includegraphics{Q1_files/figure-latex/unnamed-chunk-9-1.pdf}

\begin{Shaded}
\begin{Highlighting}[]
\NormalTok{logg\_cov }\SpecialCharTok{\%\textgreater{}\%} \FunctionTok{ggplot}\NormalTok{() }\SpecialCharTok{+} \FunctionTok{aes}\NormalTok{(}\AttributeTok{x =}\NormalTok{ coeff\_of\_var) }\SpecialCharTok{+} 
  \FunctionTok{geom\_boxplot}\NormalTok{() }\SpecialCharTok{+} 
  \FunctionTok{labs}\NormalTok{(}\AttributeTok{title=}\StringTok{\textquotesingle{}Boxplot of Distribution of Coefficient of Variation over 1000 bootstrap samples }
\StringTok{       of size 10000\textquotesingle{}}\NormalTok{, }
       \AttributeTok{x=} \StringTok{\textquotesingle{}Coefficient of Variation\textquotesingle{}}\NormalTok{)}
\end{Highlighting}
\end{Shaded}

\includegraphics{Q1_files/figure-latex/unnamed-chunk-9-2.pdf}

Question 3: Is there a correlation between the abundance of carbon on a
red giant star and the effective temperature of that star?

Extracting the relevant columns:

\begin{Shaded}
\begin{Highlighting}[]
\NormalTok{carbon\_abundance }\OtherTok{\textless{}{-}}\NormalTok{ file }\SpecialCharTok{\%\textgreater{}\%} \FunctionTok{h5read}\NormalTok{(}\StringTok{"c\_h"}\NormalTok{) }\SpecialCharTok{\%\textgreater{}\%} \FunctionTok{as\_tibble}\NormalTok{() }
\NormalTok{temp\_eff }\OtherTok{\textless{}{-}}\NormalTok{ file }\SpecialCharTok{\%\textgreater{}\%} \FunctionTok{h5read}\NormalTok{(}\StringTok{"teff"}\NormalTok{) }\SpecialCharTok{\%\textgreater{}\%} \FunctionTok{as\_tibble}\NormalTok{()}
\end{Highlighting}
\end{Shaded}

Visualizations:

\begin{Shaded}
\begin{Highlighting}[]
\FunctionTok{ggplot}\NormalTok{() }\SpecialCharTok{+} \FunctionTok{aes}\NormalTok{(}\AttributeTok{x=}\NormalTok{carbon\_abundance}\SpecialCharTok{$}\NormalTok{value, }\AttributeTok{y=}\NormalTok{temp\_eff}\SpecialCharTok{$}\NormalTok{value) }\SpecialCharTok{+} \FunctionTok{geom\_point}\NormalTok{() }\SpecialCharTok{+} 
  \FunctionTok{geom\_smooth}\NormalTok{(}\AttributeTok{method=}\StringTok{"lm"}\NormalTok{, }\AttributeTok{se=}\ConstantTok{FALSE}\NormalTok{) }\SpecialCharTok{+}
  \FunctionTok{labs}\NormalTok{(}\AttributeTok{title=}\StringTok{"Scatter Plot of Effective Temperature against Abundance of Carbon"}\NormalTok{, }
       \AttributeTok{x=}\StringTok{\textquotesingle{}Abundance of Carbon\textquotesingle{}}\NormalTok{, }\AttributeTok{y=}\StringTok{\textquotesingle{}Effective Temperature\textquotesingle{}}\NormalTok{)}
\end{Highlighting}
\end{Shaded}

\begin{verbatim}
## `geom_smooth()` using formula = 'y ~ x'
\end{verbatim}

\includegraphics{Q1_files/figure-latex/unnamed-chunk-11-1.pdf}

\end{document}
